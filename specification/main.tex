\documentclass{article}

%---------------------------------------------------------------------------------------
% Packages
%---------------------------------------------------------------------------------------

% Handles wrapping URLs as footnotes.
\usepackage[pdfborder={0 0 0},hyperfootnotes=false]{hyperref}

% Drawing of figures
\usepackage{tikz}

% Sets page margins to 1.25 inches.
\usepackage[margin=1.25in]{geometry}

% Allows for checkmarks to be included in the progress table appendix.
\usepackage{utfsym}

% Colors fonts (used in API documentation).
\usepackage{xcolor}

% Support for custom code listings.
\usepackage{listings}

%---------------------------------------------------------------------------------------
% Fonts
%---------------------------------------------------------------------------------------

% Set up bera package, which includes a roman and monospaced font. We really just want
% the monospaced font, so we load bera first and then overwrite the roman font using
% the bitstream-charter package.
\usepackage{bera}

% Set up bitstream charter as the roman font.
% Must be included after the 'bera' package to override the roman font.
\usepackage[bitstream-charter]{mathdesign}

% Import title sections package to allow changing the behavior of the section and
% subsection commands.
\usepackage{titlesec}

% Redefine subsection font to not be bolded.
\titleformat{\subsection}{\large}
{\thesubsection}{1em}{}

%---------------------------------------------------------------------------------------
% Custom listings styling
%---------------------------------------------------------------------------------------
% define some colors that will be used later.
\definecolor{comment}{RGB}{34, 139, 34}
\definecolor{string}{RGB}{164, 20, 20}

% set global listings settings.
\lstset{
    basicstyle=\normalsize\ttfamily,
    breaklines=true,
    frame=trBL,
    frameround=tttt
}

% define a custom highlighting scheme for our api code blocks. This resembles JSON but
% allows for comments (annotated by two slashes).
\lstdefinelanguage{api}{
    numbers=left,
    numberstyle=\small,
    morestring=[b]",
    morecomment=[l]{//},
    stringstyle=\color{string},
    commentstyle=\color{comment} \emph
}

%---------------------------------------------------------------------------------------
% Frontmatter
%---------------------------------------------------------------------------------------

\title{
    Atlas Specification \\
    {\Large \emph{Version 1.0 (In Progress)}}
}
\author{
    \begin{tabular}[t]{cc}
        Clay McLeod$^1$ & Michael Macias$^1$ \\
        \texttt{clay.mcleod@stjude.org} & \texttt{michael.macias@stjude.org}
    \end{tabular}
}

\date{
    $^1$St. Jude Children's Research Hospital \\[2ex]
    \today
}

%---------------------------------------------------------------------------------------
% New commands to make our lives easier
%---------------------------------------------------------------------------------------
\newcommand{\Sample}{\texttt{Sample}}
\newcommand{\Samples}{\texttt{Sample}s}

%---------------------------------------------------------------------------------------
% Begin the document main body
%---------------------------------------------------------------------------------------

\begin{document}

\maketitle
\tableofcontents

\section{Introduction}
Bioinformatics regularly requires storage and processing of large, numerical
matrices representing quantitative measurements across a cohort. These matrices,
which are often spare in nature, cannot generally be processed trivially within
memory. Further, the number of samples being processed and visualized is growing
rapidly, quickly exposing the limitations of existing tooling. We introduce the
Atlas quantification specification and associated reference implementation
(\url{https://github.com/stjude-rust-labs/atlas}) to address the unmet needs
described above.

\section{Goals}

The stated goals of the Atlas project are four-fold:

\begin{enumerate}
    \item Create an API-based specification for storing, querying, manipulating,
          and rendering of large quantitative matrices.
    \item Support commonly used quantitative data types and associated analyses,
          including (in this order of priority) RNA-Seq expression data, DNA and RNA
          microarray data, and methylation data.
    \item Create a performant reference implementation of the aforementioned
          specification which can be used as a both a client and server.
    \item As we can, consider and implement features requested by the broader community.
\end{enumerate}

Notably, authentication and authorization are not goals of the project. We
recommend you investigate supplementary techniques to secure an Atlas server.

\section{Samples}
\label{section:samples}

Each quantitative measurement within Atlas must belong to a \Sample. Storage and
querying of \Samples{ }within Atlas is intended to be straightforward and flexible, not
capturing the complexities of real-world sample collection and aggregation. \Samples{
}are simply assigned a name at creation time; an ID number and appropriate timestamps
are added alongside the name. \textit{Note: if you need to organize samples by cohort or
    otherwise, consider using a sample prefix (e.g. \texttt{Cohort/Sample}).} Table
\ref{table:sample-fields} describes the fields associated with a \Sample{ }in more
detail.

\begin{table}[ht]
    \centering
    \begin{tabular}{|l|r|l|}
        \hline
        \textbf{Field Name}  & \textbf{Field Type} & \textbf{Description}                              \\
        \hline
        \texttt{id}          & \texttt{i32}        & Primary assigned ID for this sample within Atlas. \\
        \texttt{name}        & \texttt{String}     & Name of the sample.                               \\
        \texttt{created\_at} & \texttt{Timestampz} & Time that the sample was created in Atlas.        \\
        \hline
    \end{tabular}
    \label{table:sample-fields}
    \caption{Detailed description of \Sample{ }fields.}
\end{table}

\section{API Reference}
\subsection{\texttt{/api/samples} — Listing Samples}

This API method is useful to interrogate what samples the Atlas server currently has
data for.  You can take a look at Section \ref{section:samples} for more information
about what is contained within a \texttt{Sample} object.

\vspace{1.5em}
\begin{lstlisting}[language=api]
// GET /api/samples

{
    "results": Sample[], // samples currently contained in Atlas.
    "length":  usize     // length of the results.
}
\end{lstlisting}


\subsection{\texttt{/api/samples/\color{purple}:name} — Querying a Sample}

This API method returns information about a particular sample stored in the Atlas
server.
% \subsection{\texttt{/api/counts/\color{purple}:id} — Viewing Counts}

\appendix

\section{Reference Implementation Status}

\newcommand{\cm}{\usym{1F5F8}}
\begin{center}
    \begin{tabular}{|p{150pt}|l|c|c|c|}
        \hline
        \textbf{Feature Name} & \textbf{API Method}                       & \textbf{Completed} & \textbf{Released} & \textbf{Released Version} \\ [0.5ex]
        \hline
        Listing Samples       & \texttt{/api/samples}                     & \cm                &                   & \emph{N/A}                \\
        Querying a Sample     & \texttt{/api/samples/\color{purple}:name} & \cm                &                   & \emph{N/A}                \\
        \hline
    \end{tabular}
\end{center}

\end{document}